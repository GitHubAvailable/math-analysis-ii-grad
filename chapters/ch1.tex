\documentclass[class=book, crop=false]{standalone}
\usepackage{xcolor}
\usepackage{utils}
\usepackage{import}

\begin{document}
    \section{Basic Concepts, Measurable Spaces}
        \subsection{Limit of Sequence of Sets}
        \begin{definition}[Upper Limit Set]
            $\varlimsup_{n \rightarrow \infty} A_n = \bigcap^{\infty}_{i = 1} \bigcup^{\infty}_{n = i} A_j$.
        \end{definition}
        
        \begin{definition}[Lower Limit Set]
            $\varliminf_{n \rightarrow \infty} A_n = \bigcup^{\infty}_{i = 1} \bigcap^{\infty}_{n = i} A_j$.
        \end{definition}
        
        \begin{definition}[Indicator Function]
            Let $S \subseteq U$ be a set and $x \in U$. The indicator function of $S$ $\indic_S(x) : U \rightarrow \{0, 1\}$ is defined to return $1$ if $x \in A$ and $0$ otherwise.
        \end{definition}
        
        \noindent The following are alternate definitions of set limit using indicators.
        \begin{theorem}
            $\varlimsup_{n \rightarrow \infty} A_n = \{x : \varlimsup_{n \rightarrow \infty} \indic_{A_n}(x) = 1\}$.
        \end{theorem}
        
        \begin{theorem}
            $\varliminf_{n \rightarrow \infty} A_n = \{x : \varliminf_{n \rightarrow \infty} \indic_{A_n}(x) = 1\}$.
        \end{theorem}

        % Limit of sequence of sets
        \begin{theorem}[Existence of Limit for $\{A_n\}^{\infty}_{n = 1}$]
            If the upper limit set and the lower limit set are equal to each other, then
            \begin{equation*}
                \lim_{n \rightarrow \infty} A_n := \varlimsup_{n \rightarrow \infty} A_n = \varliminf_{n \rightarrow \infty} A_n
            \end{equation*}
        \end{theorem}
        
        \begin{corollary}
            If $\lim_{n \rightarrow \infty} \indic_{A_n}(x)$ exists for all $x$, then
            \begin{equation*}
                \lim_{n \rightarrow \infty} A_n = \{x : \lim_{n \rightarrow \infty} \indic_{A_n}(x) = 1\}.
            \end{equation*}
            Otherwise, the set sequence diverges.
        \end{corollary}
        
        \begin{theorem}
            $\varliminf_{n \rightarrow \infty} A_n \subseteq \varlimsup_{n \rightarrow \infty} A_n$.
        \end{theorem}


        \subsection{Measurable Spaces}
        \begin{definition}[Measure]
            Let $X$ be a set. A measure on $X$ is a function that assigns a non-negative real number to subsets of $X$.
        \end{definition}

        \begin{definition}[Algebra]
            Let $X$ be a nonempty set and $\mathscr{A}$ be a family of subsets of $X$. $\mathscr{A}$ is an algebra of $X$ if it satsifies all the following:
            \begin{enumerate}
                \item $X, \emptyset \in \mathscr{A}$;
                \item $\mathscr{A}$ is closed under union and complement.
            \end{enumerate}
        \end{definition}
        
        \begin{definition}[$\sigma$-Algebra] \label{def:sigma-algebra}
            Let $X$ be a set and $\mathcal{A} \subseteq \pwset(X)$. $\mathcal{A}$ is a $\sigma$-algebra if all the following are satisfied:
            \begin{enumerate}
                \item $X \in \mathcal{A}$.
                \item Closed under complementation: $S \in \mathcal{A} \implies S^C \in \mathcal{A}$.
                \item Closed under countable unions: $A_1, A_2, A_3, \in \mathcal{A} \implies \bigcup^{\infty}_{i = 1} A_n \in \mathcal{A}$.
            \end{enumerate}
        \end{definition}
        
        \begin{definition}[Measurable Space]
            Let $X$ be a set and $\mathcal{A} \subseteq \pwset(X)$ be a $\sigma$-algebra. The ordered pair $(X, \mathcal{A})$ is called a measurable space.
        \end{definition}
        
        \begin{theorem}
            A measurable space is closed under countable intersections.
        \end{theorem}
        \begin{proof}
            Apply DeMorgan's Laws to the last condition of $\sigma$-algebra in \Cref{def:sigma-algebra}.
        \end{proof}

        \begin{theorem}
            Let $X$ be a set. $\{X, \emptyset\}$ is the smallest $\sigma$-algebra on $X$, and $\pwset(X)$ is the largest.
        \end{theorem}

        \begin{remark}
            The $\sigma$-algebra on a set $X$ is closed under basic set operations.
        \end{remark}

        \begin{definition}[Measurable Mapping]
            Let $(X, \mathcal{A})$ and $(Y, \mathcal{B})$ be measurable spaces. A mapping $f : X \rightarrow Y$ is measurable if for every $B \in \mathcal{B}$, $f^{-1}(B) \in \mathcal{A}$.
        \end{definition}

        \subsection{Presentation Question}
        \begin{question}
            Let $(X, \mathcal{A})$ and $(Y, \mathcal{B})$ be measurable spaces. Suppose $\mathcal{B}$ is generated by a family of subsets $\mathscr{F} \subseteq \mathscr{P}(Y)$, that is, $\mathcal{B} = \sigma(\mathscr{F})$. Then $f: X \rightarrow Y$ is $(\mathcal{A}, \mathcal{B})$ measurable if and only if $f^{-1}(F) \in \mathcal{A}$ for every $F \in \mathscr{F}$.
        \end{question}
        \begin{remark}
            ``Generated by" means ``the smallest $\sigma$-algebra containing". $B = \sigma(\mathscr{F})$ means $\mathcal{B}$ is the smallest $\sigma$-algebra such that $\mathscr{F}$ is its subset.
        \end{remark}
        \begin{proof}
            \textit{Proof of Sufficiency.} Since $f : X \rightarrow Y$ is $(\mathcal{A}, \mathcal{B})$ measurable, $f^{-1}(F) \in \mathcal{A}$ for every $F \in \mathscr{F} \subseteq \mathcal{B}$ by definition.
    
            \noindent \textit{Proof of Necessity.} Let $\mathcal{M} := \{S \subseteq Y : f^{-1}(S) \in \mathcal{A}\}$. Observe that for any $S \subseteq Y$,
            \begin{equation*}
                f^{-1}(S^C) = (f^{-1}(S))^{C} \in \mathcal{A} \implies S^C \in \mathcal{M}. 
            \end{equation*}
            And if $B_i \in \mathcal{M}$, then $f^{-1}(B_i) \in \mathcal{A}$ and
            \begin{equation*}
                f^{-1}\left(\bigcup^{\infty}_{i = 1} B_i\right) = \bigcup^{\infty}_{i = 1} f^{-1}(B_i) \in \mathcal{A} \implies \bigcup^{\infty}_{i = 1} B_i \in \mathcal{A}.
            \end{equation*}
            So $\mathcal{M}$ is a $\sigma$-algebra and $f$ is $(\mathcal{A}, \mathcal{M})$ measurable. It follows that
            \begin{equation*}
                \mathcal{B} = \sigma(\mathscr{F}) \subseteq \mathcal{M}.
            \end{equation*}
            Thus, $f^{-1}(F) \in \mathcal{A}$ for any $F \in \mathcal{B}$ and $f$ is $(\mathcal{A}, \mathcal{B})$ measurable.
        \end{proof}


    \section{Lebesgue Outer Measure, Lebesgue Measurable Set}
        \subsection{Lebesgue Outer Measure}
        \begin{definition}[Measure Space]
            Let $X$ set and $\mathcal{A}$ be a $\sigma$-algebra about $X$. Let $\mu : \mathcal{A} \rightarrow [0, +\infty)$. Then, $(X, \mathcal{A}, \mu)$ forms a measure space if $\mu$ satisfies invariance in translation and additivity.
        \end{definition}

        \begin{definition}[Lebesgue Outer Measure for $\R$]
            Let $E \subseteq \R$. Let $\{I_n\}^{\infty}_{n = 1}$ be a sequence of open, bounded intervals over $\R$. Denote the length of each $I_i$ as $|I_i|$. The outer measure $m^*(E)$ is defined as
            \begin{equation*}
                m^*(E) = \inf\left\{\sum^{\infty}_{i = 1} |I_i| : E \subseteq \bigcup^{\infty}_{i = 1} I_i\right\}.
            \end{equation*}
        \end{definition}

        \begin{theorem}
            The Lebesgue outer measure of any countable set is $0$.
        \end{theorem}

        \begin{theorem}[Monotonicity]
            If $A \subseteq B$, then $m^*(A) \leq m^*(B)$.
        \end{theorem}
        \begin{proof}
            Observe that
            \begin{equation*}
                \left\{\sum^{\infty}_{i = 1} |I_i| : A \subseteq \bigcup^{\infty}_{i = 1} I_i\right\} \supseteq \left\{\sum^{\infty}_{i = 1} |I_i| : B \subseteq \bigcup^{\infty}_{i = 1} I_i\right\}.
            \end{equation*}
        \end{proof}
        \begin{remark}
            The infimum of a set is at most the infimum of its subset.
        \end{remark}

        \begin{theorem}
            If $I$ is an interval, then $m^* = |I|$.
        \end{theorem}

        \begin{theorem}[Subadditivity]
            If $E_k \subseteq \R$, $k = 1, 2, \dots$, then
            \begin{equation*}
                m^*\left(\bigcup^{\infty}_{k = 1} E_k\right) \leq \sum^{\infty}_{k = 1} m^*(E_k).
            \end{equation*}
        \end{theorem}
        \begin{remark}
            There are some strange sets that don't satisfy additivity.
        \end{remark}

        \begin{theorem}[Translation Invariance]
            Let $x \in \R$. Then $m^*(E + x) = m^*(x)$.
        \end{theorem}

        \subsection{Lebesgue Measure}
        \begin{definition}[Carath\'eodory Condition]
            Set $E \subseteq \R$ is said to be Lebesgue measurable if for any set $A \subseteq \R$,
            \begin{equation*}
                m^*(A) = m^*(A \cap E) + m^*(A \cap E^C).
            \end{equation*}
        \end{definition}

        \begin{definition}[Lebesgue Measure]
            The $\sigma$-algebra of set of Lebesgue measurable sets $\mathscr{L}$ is \textit{Lebesgue $\sigma$-algebra}. Lebesgue measure $m(E) = m^*(E)$ for $E \in \mathscr{L}$. $(\R, \mathscr{L})$ is called \textit{Lebesgue measurable space}. $(\R, \mathscr{L}, m)$ is called the \textit{Lebesgue measure space}.
        \end{definition}
        
        \begin{theorem}
            $E \in \mathscr{L} \Leftrightarrow E^C \in \mathscr{L}$.
        \end{theorem}

        \begin{theorem}
            $m^*(E) = 0 \implies E \in \mathscr{L}$.
        \end{theorem}
        \begin{remark}
            Countable intersection of open sets $G_{\delta}$ and countable union of closed sets $F_{\sigma}$ are Lebesgue measurable.
        \end{remark}

        \begin{theorem}
            $\Leb$ is a $\sigma$-algebra.
        \end{theorem}


        \subsection{Presentation Question}
        \begin{question}
            Let $A$ be the set of rational numbers in $[0, 1]$. Prove that $m^*(A) = 0$ and $m^*([0, 1] \backslash A) = 1$.
        \end{question}
        \begin{proof}
            Let $x \in A$. Then, if $x \neq 0$ there exists $v = (m, n) \in \Z^2$ such that $m, n$ are coprime and $\frac{m}{n} = x$. Then, there exists a mapping $h: A \rightarrow \Z^2$ such that
            \begin{equation*}
                h(x) = \begin{cases}
                    (m, n) & (x \in A \cap (0, 1)) \\
                    (1, 1) & (x = 1) \\
                    (0, 1) & (x = 0)
                \end{cases}
            \end{equation*}
            Since $\Z$ is countable, $\Z^{2}$ is countable. It follows that $A$ is countable and $m^*(A) = 0$. \\
            By monotonicity, $m^*([0, 1] \backslash A) \leq m^*([0, 1]) = 1$ since $([0, 1] \backslash A) \subseteq [0, 1]$. On the other hand, by subadditivity,
            \begin{equation*}
                1 = m^*([0, 1]) \leq m^*(A) + m^*([0, 1] \backslash A) = 0 + m^*([0, 1] \backslash A).
            \end{equation*}
            So $m^*([0, 1] \backslash A) = 1$.
        \end{proof}


    \section{Lebesgue Measure, Properties, and Characterization}
    \subsection{Properties of Lebesgue Measure}
    \begin{theorem}[$\sigma$-Additivity]
        If $E_i$, $i \in \N$, are disjoint Lebesgue measurable sets, then
        \begin{equation*}
            m\left(\bigcup^{\infty}_{i = 1} E_i\right) = \sum^{\infty}_{i = 1} m(E_i).
        \end{equation*}
    \end{theorem}
    
    \begin{theorem}[Subtraction]
        Let $E, F \in \mathscr{L}$, then
        \begin{equation*}
            F \subseteq E, m(F) < \infty \implies m(E \backslash F) = m(E) - m(F).
        \end{equation*}
    \end{theorem}
    
    \begin{definition}[Borel Algebra]
        The Borel $\sigma$-algebra is the smallest algebra containing all open sets that is, it is the $\sigma$-algebra generated by open sets with the operations of infinite union and complement, denoted by $\mathscr{B}(\R)$, which is a subset of $\mathscr{L}$.
    \end{definition}

    \begin{theorem}[Continuity]
        If $E_i \in \mathscr{L}$, $i \in \N$, are increasing, then
        \begin{equation*}
            m\left(\bigcup^{\infty}_{i = 1} E_i\right) = \lim_{i \rightarrow \infty} m(E_i).
        \end{equation*}
        If $E_i \in \mathscr{L}$, $i \in \N$, are decreasing, and $m(E_{i_1}) < \infty$ for some $E_{i_1}$, then
        \begin{equation*}
            m\left(\bigcap^{\infty}_{i = 1} E_i\right) = \lim_{i \rightarrow \infty} m(E_i).
        \end{equation*}
    \end{theorem}

    \begin{theorem}[The Borel-Cantelli Lemma]
        Let $E_i$ be a sequence of measurable sets. Then,
        \begin{equation*}
            \sum^{\infty}_{i = 1} m(E_i) < \infty \implies m\left(\varlimsup_{i \rightarrow \infty} E_i\right) = 0,
        \end{equation*}
        that is, almost all elements belong to finitely many $E_i$.
    \end{theorem}

    \begin{theorem}
        Let $E_i$ be a sequence of measurable sets. Then,
        \begin{equation*}
            \lim_{i \rightarrow \infty} E_i = 0 \implies m\left(\varliminf_{i \rightarrow \infty} E_i\right) = 0.
        \end{equation*}
    \end{theorem}
    
    \subsection{Characterizations}
    Characterizations describes how ``close" the Lebesgue measure is compared to the actual ``length" of a set.
    \begin{theorem}[Equivalence of being a Measurable Set]
        For any $E \subseteq \R$, the following are equivalent:
        \begin{enumerate}
            \item $E \in \mathscr{L}$;
            \item For any $\varepsilon > 0$, there exists open set $O \supseteq E$ such that $m^*(O \backslash E) < \varepsilon$;
            \item For any $\varepsilon > 0$, there exists closed set $F \subseteq E$ such that $m^*(E \backslash F) < \varepsilon$;
            \item For any $\varepsilon > 0$, there exists a closed set $F$ and open set $O$ with $F \subseteq E \subseteq O$ such that $m^*(O \backslash F) < \varepsilon$;
            \item There is a $G_{\delta}$ set $G \supseteq E$ such that $m^*(G \backslash E) = 0$;
            \item There is a $F_{\sigma}$ set $G \subseteq E$ such that $m^*(E \backslash F) = 0$.
        \end{enumerate}
    \end{theorem}

    \begin{corollary}
        Any Lebesgue measurable set is both of the following:
        \begin{enumerate}
            \item The union of a Borel set and a set of measure zero;
            \item The difference of  Borel set and a set of measure zero.
        \end{enumerate}
    \end{corollary}
    \begin{remark}
        There exists Lebesgue measurable set that is not Borel.
    \end{remark}

    \subsection{Presentation Question}
    \begin{question}
        Prove the following statements are equivalent:
        \begin{enumerate}
            \item For any $\varepsilon > 0$, there exists open set $O \supseteq E$ such that $m^*(O \backslash E) < \varepsilon$;
            \item For any $\varepsilon > 0$, there exists closed set $F \subseteq E$ such that $m^*(E \backslash F) < \varepsilon$;
            \item For any $\varepsilon > 0$, there exists a closed set $F$ and open set $O$ with $F \subseteq E \subseteq O$ such that $m^*(O \backslash F) < \varepsilon$.
        \end{enumerate}
    \end{question}

    \noindent \textbf{Key concepts:}
    \begin{defIndep}[Lebesgue Outer Measure]
        Let $E \subseteq \R$. Let $\{I_n\}^{\infty}_{n = 1}$ be a sequence of open, bounded intervals over $\R$. Denote the length of each $I_i$ as $|I_i|$. The outer measure $m^*(E)$ is defined as
        \begin{equation*}
            m^*(E) = \inf\left\{\sum^{\infty}_{i = 1} |I_i| : E \subseteq \bigcup^{\infty}_{i = 1} I_i\right\}.
        \end{equation*}
    \end{defIndep}
    \begin{theoremIndep}[Monotonicity]
        If $A \subseteq B$, then $m^*(A) \leq m^*(B)$.
    \end{theoremIndep}
    \begin{theoremIndep}[Subadditivity]
        If $E_k \subseteq \R$, $k \in \{1, 2, \dots\}$, then
        \begin{equation*}
            m^*\left(\bigcup^{\infty}_{k = 1} E_k\right) \leq \sum^{\infty}_{k = 1} m^*(E_k).
        \end{equation*}
    \end{theoremIndep}
    \begin{theoremIndep}
        The complement of an closed set is open, and vice versa.
    \end{theoremIndep}
    \begin{proof}
        Suppose Statement 1 is true. Since $E$ is arbitrary, for $E^C$, there exists open set $F^C$ with $E^C \subseteq F^C$ such that $m^*(F^C \backslash E^C) < \varepsilon$. By definition of complement, $x \in F^C \backslash E^C$ if and only if $x \in E \backslash F$. It follows that $E \backslash F = F^C \backslash E^C$. Notice that $F$ is closed since $F^C$ is open. So there exists closed set $F \subseteq E$ such that $m^*(E \backslash F) = m^*(F^C \backslash E^C) < \varepsilon$ for any $\varepsilon > 0$, thus proving Statement 2. By similar logic, Statement 1 can be proved by Statement 2. So they are equivalent. \\

        \noindent By Statement 1 and Statement 2, for any $\varepsilon > 0$, there exists a closed set $F$ and open set $O$ with $F \subseteq E \subseteq O$ such that $m^*(O \backslash E) < \frac{\varepsilon}{2}$ and $m^*(E \backslash F) < \frac{\varepsilon}{2}$. By subadditivity,
        \begin{equation*}
            m^*(O \backslash F) = m^*((O \backslash E) \cup (E \backslash F)) \leq m^*(O \backslash E) + m^*(E \backslash F) < \frac{\varepsilon}{2} + \frac{\varepsilon}{2} = \varepsilon.
        \end{equation*}

        \noindent On the other hand, suppose Statement 3 is true. For such $O$ and $F$, $F \subseteq E \subseteq O$ implies $(O \backslash E) \subseteq (O \backslash F)$ and $(E \backslash F) \subseteq (O \backslash F)$. By monotonicity,
        \begin{equation*}
            m^*(O \backslash E) \leq m^*(O \backslash F) < \varepsilon, \;\text{and}\; m^*(E \backslash F) \leq m^*(O \backslash F) < \varepsilon,
        \end{equation*}
        thus proving Statement 1. Similarly, Statement 2 can be proven by considering $(E \backslash F) \subseteq (O \subseteq F)$. So all $3$ statements are equivalent.
    \end{proof}
\end{document}