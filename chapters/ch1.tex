\documentclass[class=book, crop=false]{standalone}
\usepackage{xcolor}
\usepackage{utils}
\usepackage{import}

\begin{document}
    \section{Basic Concepts, Measurable Spaces}
        \subsection{Limit of Sequence of Sets}
        \begin{definition}[Upper Limit Set]
            $\uplim_{n \rightarrow \infty} A_n = \bigcap^{\infty}_{i = 1} \bigcup^{\infty}_{n = i} A_j$.
        \end{definition}
        
        \begin{definition}[Lower Limit Set]
            $\lowlim_{n \rightarrow \infty} A_n = \bigcup^{\infty}_{i = 1} \bigcap^{\infty}_{n = i} A_j$.
        \end{definition}
        
        \begin{definition}[Indicator Function]
            Let $S \subset U$ be a set and $x \in U$. The indicator function of $S$ $\indic_S(x) : U \rightarrow \{0, 1\}$ is defined to return $1$ if $x \in A$ and $0$ otherwise.
        \end{definition}
        
        \noindent The following are alternate definitions of set limit using indicators.
        \begin{theorem}
            $\uplim_{n \rightarrow \infty} A_n = \{x : \uplim_{n \rightarrow \infty} \indic_{A_n}(x) = 1\}$.
        \end{theorem}
        
        \begin{theorem}
            $\lowlim_{n \rightarrow \infty} A_n = \{x : \lowlim_{n \rightarrow \infty} \indic_{A_n}(x) = 1\}$.
        \end{theorem}

        % Limit of sequence of sets
        \begin{theorem}[Existence of Limit for $\{A_n\}^{\infty}_{n = 1}$]
            If the upper limit set and the lower limit set are equal to each other, then
            \begin{equation*}
                \lim_{n \rightarrow \infty} A_n := \uplim_{n \rightarrow \infty} A_n = \lowlim_{n \rightarrow \infty} A_n
            \end{equation*}
        \end{theorem}
        
        \begin{corollary}
            If $\lim_{n \rightarrow \infty} \indic_{A_n}(x)$ exists for all $x$, then
            \begin{equation*}
                \lim_{n \rightarrow \infty} A_n = \{x : \lim_{n \rightarrow \infty} \indic_{A_n}(x) = 1\}.
            \end{equation*}
            Otherwise, the set sequence diverges.
        \end{corollary}
        
        \begin{theorem}
            $\lowlim_{n \rightarrow \infty} A_n \subset \uplim_{n \rightarrow \infty} A_n$.
        \end{theorem}


        \subsection{Measurable Spaces}
        \begin{definition}[Measure]
            Let $X$ be a set. A measure on $X$ is a function that assigns a non-negative real number to subsets of $X$.
        \end{definition}
        
        \begin{definition}[$\sigma$-Algebra] \label{def:sigma-algebra}
            Let $X$ be a set and $\mathcal{A} \subset \pwset(X)$. $\mathcal{A}$ is a $\sigma$-algebra if all the following are satisfied:
            \begin{enumerate}
                \item $X \in \mathcal{A}$.
                \item Closed under complementation: $S \in \mathcal{A} \implies S^C \in \mathcal{A}$.
                \item Closed under countable unions: $A_1, A_2, A_3, \in \mathcal{A} \implies \bigcup^{\infty}_{i = 1} A_n \in \mathcal{A}$.
            \end{enumerate}
        \end{definition}
        
        \begin{definition}[Measurable Space]
            Let $X$ be a set and $\mathcal{A} \subset \pwset(X)$ be a $\sigma$-algebra. The ordered pair $(X, \mathcal{A})$ is called a measurable space.
        \end{definition}
        
        \begin{theorem}
            A measurable space is closed under countable intersections.
        \end{theorem}
        \begin{proof}
            Apply DeMorgan's Laws to the last condition of $\sigma$-algebra in \Cref{def:sigma-algebra}.
        \end{proof}

        \begin{theorem}
            Let $X$ be a set. $\{X, \emptyset\}$ is the smallest $\sigma$-algebra on $X$, and $\pwset(X)$ is the largest.
        \end{theorem}

        \begin{remark}
            The $\sigma$-algebra on a set $X$ is closed under basic set operations.
        \end{remark}

        \begin{definition}[Measurable Mapping]
            Let $(X, \mathcal{A})$ and $(Y, \mathcal{B})$ be measurable spaces. A mapping $f : X \rightarrow Y$ is measurable if for every $B \in \mathcal{B}$, $f^{-1}(B) \in \mathcal{A}$.
        \end{definition}

        \subsection{Presentation Question}
        \begin{question}
            Let $(X, \mathcal{A})$ and $(Y, \mathcal{B})$ be measurable spaces. Suppose $\mathcal{B}$ is generated by a family of subsets $\mathscr{F} \subset \mathscr{P}(Y)$. Then $f: X \rightarrow Y$ is $(\mathcal{A}, \mathcal{B})$ measurable if and only if $f^{-1}(F) \in \mathcal{A}$ for every $F \in \mathscr{F}$.
        \end{question}
        \begin{proof}
            \textit{Proof of Sufficiency.} Let $F \in \mathscr{F}$. It follows that $F \in \mathcal{B}$ since the latter is generated by $\mathscr{F}$. Since $f : X \rightarrow Y$ is $(\mathcal{A}, \mathcal{B})$ measurable, there must have $f^{-1}(F) \in \mathcal{A}$.
    
            \noindent \texttt{Proof of Necessity.} Let $F \in \mathscr{F}$. Since the latter is generated by $\mathscr{F}$, any element in $\mathcal{B}$ is an element of $F$. So $F \in \mathcal{B}$. By assumption, $f^{-1}(F) \in \mathcal{A}$. This implies for every element $B \in \mathcal{B}$ there exists $A = f^{-1}(B) \in \mathcal{A}$. Thus, by definition, $f: X \rightarrow Y$ is $(\mathcal{A}, \mathcal{B})$ measurable.
        \end{proof}
\end{document}