\documentclass[class=book, crop=false]{standalone}
\usepackage{xcolor}
\usepackage{utils}
\usepackage{import}

\begin{document}
    \section{Measurable Functions}
        \subsection{Measurable Functions}
        \begin{definition}[Measurable Functions]
            Let $(X, \mathcal{M}), (Y, \mathcal{N})$ be measurable spaces. $f : X \rightarrow Y$ is measurable if $f^{-1}(F) \in \mathscr{M}$ for any $F \in \mathcal{N}$.
        \end{definition}

        \begin{theorem}
            $f$ is measurable if and only if any of the following is true:
            \begin{enumerate}
                \item $f^{-1}([a, \infty)) \in \mathcal{M}$ for any $a \in \R$;
                \item $f^{-1}((-\infty, a)) \in \mathcal{M}$ for any $a \in \R$;
                \item $f^{-1}((a, b)) \in \mathcal{M}$ for any $a, b \in \R$;
                \item $f^{-1}([a, b]) \in \mathcal{M}$ for any $a, b \in \R$;
                \item $f^{-1}(O) \in \mathcal{M}$, for any open set $O \subseteq \R$;
                \item $f^{-1}(B) \in \mathcal{M}$, for any Borel set $B \subseteq \R$.
            \end{enumerate}
        \end{theorem}
        \begin{remark}
            For the first two, open or close on the finite side does not matter.
        \end{remark}
        \begin{remark}
            If the image of $f$ is $\overline{\R}$ (i.e., $[-\infty, +\infty]$), $f^{-1}(\{-\infty\}, f^{-1}(\{+\infty\})$ needs to fall within $\mathcal{M}$.
        \end{remark}

        \begin{theorem}
            $f : E \rightarrow [-\infty, +\infty]$ is measurable if $f^{-1}((a, +\infty])$ is measurable for any $a \in \R$.
        \end{theorem}

        \begin{theorem}
            Continuous functions and monotone functions are measurable.
        \end{theorem}
        \begin{corollary}
            If $f$ is measurable, $f^{-1}$ is measurable.
        \end{corollary}
        \begin{corollary}
            If $f : E \rightarrow \R$ is measurable, then $f^{-1}(a) = \{x \in E : f(x) = a\}$ is measurable.
        \end{corollary}
        \begin{remark}
            The converse is not true. $f^{-1}(a) = \{x \in E : f(x) = a\}$ is measurable does not imply $f$ is measurable.
        \end{remark}

        \begin{theorem}
            The indicator function is measurable if and only if the set $E$ is measurable.
        \end{theorem}

        \begin{theorem}
            If $f : E \rightarrow \R$ is measurable, then $\delta|_{E_0}$ is measurable when $E_0 \subseteq E$ is measurable.
        \end{theorem}

        \begin{theorem}
            Let $E = E_1 \cup E_2$. Suppose $E_1$ and $E_2$ are measurable. Then $f : E \rightarrow \R$ is measurable if and only if $f|_{E_1}$ and $f|_{E_2}$ are measurable.
        \end{theorem}

        \begin{theorem}
            If $f, g$ are measurable, $af + bg$, $f \cdot g$ and $\frac{f}{g} \; (g \neq 0)$ are measurable. $\max\{f, g\}, \min\{f, g\}$ and $|f|$ is also measurable.
        \end{theorem}

        \begin{theorem}
            If $g : E \rightarrow \R$ is measurable and $f : \R \rightarrow \R$ is continuous, then $f(g) : E \rightarrow \R$ is measurable and
            \begin{equation*}
                (f(g))^{-1}((a, +\infty)) = g^{-1}(f^{-1}((a, +\infty)))
            \end{equation*}
        \end{theorem}

        \begin{definition}[Borel Set on Extended Real Numbers]
            $\mathscr{B}(\overline{\R}) := \{E \subseteq \R : E \cap \R \in \mathscr{B}(\R)\}$.
        \end{definition}

        \subsection{Convergence}
        \begin{definition}[Pointwise Convergence]
            $f_n \rightarrow f$ pointwise if for $x_0 \in E$,
            \begin{equation*}
                \lim_{n \rightarrow \infty} f_n(x_0) = f(x_0).
            \end{equation*}
        \end{definition}
        \begin{remark}
            Let $E_n(\varepsilon) = \{x : |f_n(x) - f(x)| \geq \varepsilon\}$. Then converging pointwise is equivalent to $\varlimsup_{n \rightarrow \infty} E_n(\varepsilon) = \emptyset$.
        \end{remark}

        \begin{definition}[Uniform Convergence]
            $f_n \rightarrow f$ uniformly if
            \begin{equation*}
                \lim_{n \rightarrow \infty} \sup_{x \in E} |f_n(x) - f(x)| = 0.
            \end{equation*}
        \end{definition}
        \begin{remark}
            Converging uniformly is equivalent to for any $\varepsilon > 0$, there exists $N$ such that when $n > N$, $E_n(\varepsilon) = \emptyset$.
        \end{remark}

        \begin{definition}[Almost Everywhere Convergence]
            $f_n \rightarrow f$ almost everywhere (a.e.) if $f_n \rightarrow f$ pointwise in $E \backslash E_0$ where $\mu(E_0) = 0$.
        \end{definition}
        \begin{remark}
            Almost everywhere convergence is equivalent to any $\varepsilon > 0$, $\mu(\varlimsup_{n \rightarrow \infty} E_n(\varepsilon) = 0$.
        \end{remark}

        \begin{definition}[Convergence in Measure]
            Let $(X, \mathcal{M}, \mu)$ be a measure space and $f_n, f : E \rightarrow \R$ be measurable. Then $f_n \rightarrow f$ in measure if
            \begin{equation*}
                \lim_{n \rightarrow \infty} \mu(\{x \in E : |f_n(x) - f(x)| \geq \varepsilon\}) = 0.
            \end{equation*}
        \end{definition}
        \begin{remark}
            Note that the set above is the set containing the point at which $f_n$ is not convergent.
        \end{remark}
        \begin{remark}
            It is not related to convergence almost everywhere.
        \end{remark}

        \begin{theorem}
            Let $f_n$ be measurable and $\mu$ a complete measure. If $f_n \rightarrow f$ almost everywhere then $f$ is measurable.
        \end{theorem}
        \begin{corollary}
            $f_n \rightarrow f$ pointwise implies $f$ is measurable.
        \end{corollary}
        
        \subsection{Presentation Question}
        \begin{question}
            Let $f, g : E \rightarrow \overline{\R}$ be measurable, and $E_0 = \{x \in E : f(x) = g(x) = \pm\infty\}$. Show that
            \begin{equation*}
                h(x) = \begin{cases}
                    f(x) - g(x) & x \in E \backslash E_0 \\
                    1 & x \in E_0
                \end{cases},
            \end{equation*}
            is measurable.
        \end{question}

        \begin{proof}
            Since $f, g$ are measurable, $h(x)$ is measurable on $E \backslash E_0$, and we also have
            \begin{equation*}
                f^{-1}(\{\pm \infty\}), g^{-1}(\{\pm\infty\}) \in \mathcal{M} \implies E_0 = f^{-1}(\{\pm \infty\}) \cap g^{-1}(\{\pm\infty\}) \in \mathcal{M}.
            \end{equation*}
            So $h^{-1}(h(E_0)) = E_0 \in \mathcal{M}$. $h$ is measurable on $E_0$. Therfore, $h$ is measurable.
        \end{proof}


    \section{Convergence in Measure, Egoroff’s Theorem, Lusin’s Theorem}
        \subsection{Convergence}
        \begin{theorem}
            If $f_n \rightarrow f$ in measure on $E$, then there is a subsequence $f_{n_k} \rightarrow f$ almost everywhere.
        \end{theorem}
        \begin{proof}
            $f_n \rightarrow f$ in measure. This implies
            \begin{equation*}
                \lim_{n \rightarrow \infty} \mu\left(E_n\left(\frac{1}{k}\right)\right) = 0 \implies \mu\left(E_{n_k}\left(\frac{1}{k}\right)\right) < \frac{1}{2^k}.
            \end{equation*}
            Then
            \begin{equation*}
                \sum^{\infty}_{k = 1} \mu\left(E_{n_k}\left(\frac{1}{k}\right)\right) < 1.
            \end{equation*}
            By Borel-Cantelli Lemma, we have
            \begin{align*}
                \implies \mu\left(\varlimsup_{k \rightarrow \infty} E_{n_k}\left(\frac{1}{k}\right)\right) &= 0 \\
                \implies f_{n_k}(x) &\rightarrow f(x) \quad a.e.
            \end{align*}
        \end{proof}

        \begin{theorem}
            If $f_n \rightarrow f$ almost everywhere and $\mu(E) < \infty$, then $f_n \rightarrow f$ in measure.
        \end{theorem}
        \begin{proof}
            Suppose it is pointwise convergent. Then
            \begin{align*}
                \varlimsup_{n \rightarrow \infty} E_n(\varepsilon) &= \emptyset. \\
                \implies \mu\left(\varlimsup_{n \rightarrow \infty} E_n(\varepsilon)\right) &= 0.
            \end{align*}
            If $\mu(E) < \infty$, then
            \begin{equation*}
                \varlimsup_{n \rightarrow \infty} \mu( E_n(\varepsilon)) \leq \mu\left(\varlimsup_{n \rightarrow \infty} E_n(\varepsilon)\right) = 0.
            \end{equation*}
        \end{proof}

        \begin{theorem}[Egroff's Theorem]
            Suppose $\mu(E) < \infty$, and $f_n \rightarrow f$ almost everywhere. Then for any $\varepsilon$ there is a measurable set $F \subseteq E$ s.t. $\mu(E \backslash F) < \varepsilon$ and $f_n \rightarrow f$ uniformly on $F$.
        \end{theorem}
        \begin{proof}
            Observe that
            \begin{align*}
                \varlimsup_{n \rightarrow \infty} E_n(\varepsilon) &= \emptyset \\
                \bigcap^{\infty}_{n = 1} \bigcup^{\infty}_{m = n} E_m(\varepsilon) &= \emptyset \\
                E'_n(\varepsilon) &= \bigcup^{\infty}_{m = n} E_m(\varepsilon) \rightarrow \emptyset \\
                \mu(E'_n(\varepsilon)) &\rightarrow 0 \\
                \mu\left(E'_{n_k}\left(\frac{1}{k}\right)\right) &< \frac{\varepsilon}{2^k}, \quad k \in \N.
            \end{align*}
            This implies the ``bad points" vanish as $n \rightarrow \infty$. It follows that
            \begin{align*}
                F' &= \bigcup^{\infty}_{k = 1} E'_{n_k}\left(\frac{1}{k}\right) \quad F = E \backslash F' \\
                F &= \bigcap^{\infty}_{k = 1} \left(E \backslash E'_{n_k}\left(\frac{1}{k}\right)\right) = \bigcap^{\infty}_{k = 1} \bigcap^{\infty}_{n = n_k} \left\{x \in E : |f_n(x) - f| < \frac{1}{k}\right\}.
            \end{align*}
            Define
            \begin{align*}
                f_n(x) = \begin{cases}
                    1 & x \in I_n \\
                    0 & x \in [0, 1] \backslash I_n
                \end{cases}
            \end{align*}
            For $x \in F$, $|f_n(x) - f(x)| < \frac{1}{k}$ where $n \geq n_k, k \in \N$.
        \end{proof}

        \begin{theorem}[Lusin's Theorem]
            Let $f : E \rightarrow \R$ be a Lebesgue measurable function. For every $\varepsilon > 0$, there is a closed set $F \subseteq E$ such that $m(E \backslash F) < \varepsilon$ and $f|_F : F \rightarrow \R$ is continuous.
        \end{theorem}

        \begin{theorem}[Lusin's Theorem (Reformulation)]
            Let $f : E \rightarrow \R$ be a Lebesgue measurable function. For every $\varepsilon > 0$, there is a continuous function $g : \R \rightarrow \R$ and a closed set $F \subseteq E$ such that $f = g$ on $F$ and $m(E \backslash F) < \varepsilon$.
        \end{theorem}

        \subsection{Simple Function}
        \begin{definition}[Simple Function]
            A function $\phi : E \rightarrow \R$ is a simple function if it takes finitely many values.
        \end{definition}

        \begin{theorem}[Simple Function Approximation]
            If $f : E \rightarrow [0, \infty]$ is measurable, then there is a sequence $\{\psi_n\}$ of simple functions on $E$ such that
            \begin{enumerate}
                \item $0 \leq \psi_1 \leq \psi_2 \leq \cdots \leq f$ and $\psi_n \rightarrow f$ pointwise on $E$;
                \item $\psi_n \rightarrow f$ uniformly on any subset of $E$ on which $f$ is bounded.
            \end{enumerate}
        \end{theorem}

        \begin{theorem}
            If $f : E \rightarrow \R$ is a measurable function, then there exists a sequence $\{\psi_n\}$ of simple functions on $E$ so that
            \begin{enumerate}
                \item $0 \leq |\varphi_1| \leq |\varphi_2| \leq \cdots \leq |f|$, and $\varphi_n \rightarrow f$ pointwise on $E$;
                \item $\varphi_n \rightarrow f$ uniformly on any subset of $E$ on which $f$ is bounded.
            \end{enumerate}
        \end{theorem}

        \subsection{Presentation Question}
        \begin{question}
            Let $f_n$ and $f$ be measurable on $E$. Show that there exists $f_n \rightarrow f$ pointwise if and ony if for any $\varepsilon > 0$,
            \begin{equation*}
                \varlimsup_{n \rightarrow \infty} E_n(\varepsilon) = \emptyset.
            \end{equation*}
        \end{question}
        \begin{proof}
            \textit{Proof of Sufficiency.} Suppose there exists a sequence $f_n \rightarrow f$ pointwise. By definition, for any $\varepsilon > 0$, for any $x_0 \in E$ there exists some integer $N > 0$ such that for all $n > N$
            \begin{equation*}
                |f_n(x_0) - f(x_0)| < \varepsilon \implies x_0 \notin E_n(\varepsilon).
            \end{equation*}
            So every $x \in E$ is in finitely many $E_n$, that is,
            \begin{equation*}
                 \varlimsup_{n \rightarrow \infty} E_n(\varepsilon) = \emptyset.
            \end{equation*}

            \noindent \textit{Proof of Necessity.} Suppose for any $\varepsilon > 0$,
            \begin{equation*}
                 \varlimsup_{n \rightarrow \infty} E_n(\varepsilon) = \emptyset.
            \end{equation*}
            This implies every $x \in E$ is in finitely many $E_n$. So for any $\varepsilon > 0$, for each $x_0 \in E$ there exists positive integer $N$ such that for all $n > N$
            \begin{equation*}
                |f_n(x_0) - f(x_0)| < \varepsilon \implies \lim_{n \rightarrow \infty} f_n(x_0) = f(x_0).
            \end{equation*}
            Thus, $f_n \rightarrow f$ pointwise.
        \end{proof}
\end{document}