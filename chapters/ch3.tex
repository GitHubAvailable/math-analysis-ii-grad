\documentclass[class=book, crop=false]{standalone}
\usepackage{xcolor}
\usepackage{utils}
\usepackage{import}

\begin{document}
    \section{Measurable Functions}
        \subsection{Measurable Functions}
        \begin{definition}[Measurable Functions]
            Let $(X, \mathcal{M}), (Y, \mathcal{N})$ be measurable spaces. $f : X \rightarrow Y$ is measurable if $f^{-1}(F) \in \mathscr{M}$ for any $F \in \mathcal{N}$.
        \end{definition}

        \begin{theorem}
            $f$ is measurable if and only if any of the following is true:
            \begin{enumerate}
                \item $f^{-1}([a, \infty)) \in \mathcal{M}$ for any $a \in \R$;
                \item $f^{-1}((-\infty, a)) \in \mathcal{M}$ for any $a \in \R$;
                \item $f^{-1}((a, b)) \in \mathcal{M}$ for any $a, b \in \R$;
                \item $f^{-1}([a, b]) \in \mathcal{M}$ for any $a, b \in \R$;
                \item $f^{-1}(B) \in \mathcal{M}$, for any Borel set $B \subseteq \R$.
            \end{enumerate}
        \end{theorem}
        \begin{remark}
            For the first two, open or close on the finite side does not matter.
        \end{remark}
        \begin{remark}
            If the image of $f$ is $\overline{\R}$ (i.e., $[-\infty, +\infty]$), $f^{-1}(\{-\infty\}, f^{-1}(\{+\infty\})$ needs to fall within $\mathcal{M}$.
        \end{remark}

        \begin{theorem}
            $f : E \rightarrow [-\infty, +\infty]$ is measurable if $f^{-1}((a, +\infty])$ is measurable for any $a \in \R$.
        \end{theorem}

        \begin{theorem}
            Continuous functions and monotone functions are measurable.
        \end{theorem}
        \begin{corollary}
            If $f$ is measurable, $f^{-1}$ is measurable.
        \end{corollary}

        \begin{theorem}
            The indicator function is measurable if and only if the set $E$ is measurable.
        \end{theorem}

        \begin{theorem}
            If $f : E \rightarrow \R$ is measurable, then $\delta|_{E_0}$ is measurable when $E_0 \subseteq E$ is measurable.
        \end{theorem}

        \begin{theorem}
            Let $E = E_1 \cup E_2$. Suppose $E_1$ and $E_2$ are measurable. Then $f : E \rightarrow \R$ is measurable if and only if $f|_{E_1}$ and $f|_{E_2}$ are measurable.
        \end{theorem}

        \begin{theorem}
            If $f, g$ are measurable, $af + bg$, $f \cdot g$ and $\frac{f}{g} \; (g \neq 0)$ are measurable.
        \end{theorem}

        \begin{theorem}
            If $g : E \rightarrow \R$ is measurable and $f : \R \rightarrow \R$ is continuous, then $f(g) : E \rightarrow \R$ is measurable and
            \begin{equation*}
                (f(g))^{-1}((a, +\infty)) = g^{-1}(f^{-1}((a, +\infty)))
            \end{equation*}
        \end{theorem}

        \begin{definition}[Borel Set on Extended Real Numbers]
            $\mathscr{B}(\overline{\R}) := \{E \subseteq \R : E \cap \R \in \mathscr{B}(\R)\}$.
        \end{definition}

        \subsection{Convergence}
        \begin{definition}[Pointwise Convergence]
            $f_n \rightarrow f$ pointwise if for $x_0 \in E$,
            \begin{equation*}
                \lim_{n \rightarrow \infty} f_n(x_0) = f(x_0).
            \end{equation*}
        \end{definition}

        \begin{definition}[Uniformly Convergence]
            $f_n \rightarrow f$ uniformly if
            \begin{equation*}
                \lim_{n \rightarrow \infty} \sup_{x \in E} |f_n(x) - f(x)| = 0.
            \end{equation*}
        \end{definition}

        \begin{definition}[Almost Everywhere Convergence]
            $f_n \rightarrow f$ almost everywhere (a.e.) if $f_n \rightarrow f$ pointwise in $E \backslash E_0$ where $\mu(E_0) = 0$.
        \end{definition}

        \begin{definition}[Convergence in Measure]
            Let $(X, \mathcal{M}, \mu)$ be a measure space and $f_n, f : E \rightarrow \R$ be measurable. Then $f_n \rightarrow f$ in measure if
            \begin{equation*}
                \lim_{n \rightarrow \infty} \mu(\{x \in E : |f_n(x) - f(x)| \geq \varepsilon\}) = 0.
            \end{equation*}
        \end{definition}
        \begin{remark}
            Note that the set above is the set containing the point at which $f_n$ is not convergent.
        \end{remark}
        \begin{remark}
            It is not related to convergence almost everywhere.
        \end{remark}

        \begin{theorem}
            Let $f_n$ be measurable and $\mu$ a complete measure. If $f_n \rightarrow f$ almost everywhere then $f$ is measurable.
        \end{theorem}
        \begin{corollary}
            $f_n \rightarrow f$ pointwise implies $f$ is measurable.
        \end{corollary}
        
        \subsection{Presentation Question}
        \begin{question}
            Let $f, g : E \rightarrow \overline{\R}$ be measurable, and $E_0 = \{x \in E : f(x) = g(x) = \pm\infty\}$. Show that
            \begin{equation*}
                h(x) = \begin{cases}
                    f(x) - g(x) & x \in E \backslash E_0 \\
                    1 & x \in E_0
                \end{cases},
            \end{equation*}
            is measurable.
        \end{question}

        \begin{proof}
            Since $f, g$ are measurable, $h(x)$ is measurable on $E \backslash E_0$, and we also have
            \begin{equation*}
                f^{-1}(\{\pm \infty\}), g^{-1}(\{\pm\infty\}) \in \mathcal{M} \implies E_0 = f^{-1}(\{\pm \infty\}) \cap g^{-1}(\{\pm\infty\}) \in \mathcal{M}.
            \end{equation*}
            So $h^{-1}(h(E_0)) = E_0 \in \mathcal{M}$. $h$ is measurable on $E_0$. Therfore, $h$ is measurable.
        \end{proof}


    \section{Convergence in Measure, Egoroff’s Theorem, Lusin’s Theorem}
        \subsection{Convergence in Measure}
        
        
        \subsection{Presentation Question}
        \begin{question}
            
        \end{question}
        \begin{proof}
            
        \end{proof}
\end{document}