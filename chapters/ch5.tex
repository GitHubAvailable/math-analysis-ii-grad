\documentclass[class=book, crop=false]{standalone}
\usepackage{xcolor}
\usepackage{utils}
\usepackage{import}

\begin{document}
    \section{Product Measure, Fubini-Tonelli Theorem}
        \subsection{Product Measure Space}
        \begin{theorem}
            Let $\mathscr{A}$ be the finite union of rectangles. Then $\mathscr{A}$ is an algebra.
        \end{theorem}

        \begin{definition}[Product $\sigma$-Algebra]
            The product $\sigma$-algebra $\mathscr{M} \otimes \mathscr{N}$ is the $\sigma$-algebra generated by algebra $\mathscr{A}$.
        \end{definition}

        \begin{theorem}
            If $E \times F = \bigcup^{\infty}_{i = 1} (E_i \times F_i)$ is a disjoint union,
            \begin{equation*}
                \mu(E) \nu(F) = \sum^{\infty} \mu(E_i) \nu(F_i).
            \end{equation*}
        \end{theorem}

        \begin{definition}[Premeasure on Algebra $\mathscr{A}$]
            Define the premeasrue set function $\lambda$ on algebra $\mathscr{A}$ by
            \begin{equation*}
                \lambda\left(\bigcup^{\infty}_{i = 1} (E_i \times F_i)\right) = \sum^{\infty}_{i = 1} \mu(E_i) \nu(F_i).
            \end{equation*}
        \end{definition}
        \begin{remark}
            $\lambda$ generates an outer measure on $X \times Y$.
        \end{remark}

        \begin{definition}[Product Measure]
            The product measure $\mu \times \nu$ on $\mathscr{M} \otimes \mathscr{N}$ is defined by
            \begin{equation*}
                \mu \times \nu = \lambda^*|_{\mathscr{M} \otimes \mathscr{N}}.
            \end{equation*}
            If $\mu$ and $\nu$ are $\sigma$-finite, then $\mu \times \nu$ is $\sigma$-finite and is the unique measure on $\mathscr{M} \times \mathscr{N}$ so that
            \begin{equation*}
                (\mu \times \nu)(E \times F) = \mu(E) \nu(F).
            \end{equation*}
        \end{definition}

        \begin{definition}[Section]
            Let $A \in X \times Y$. The $x$-section $A_x$ and $y$-section $A_y$ of $A$ are
            \begin{equation*}
                A_x = \{y \in Y : (x, y) \in A\}, \quad A_y = \{x \in X : (x, y) \in A\}.
            \end{equation*}
            Let $f$ be a function on $X \times Y$. The $x$-section $f_x$ and $y$-section $f_y$ of $f$ are
            \begin{equation*}
                f_x(y) = f_y(x) = f(x, y).
            \end{equation*}
        \end{definition}

        \begin{theorem}
            If $A \in \mathscr{M} \otimes \mathscr{N}$, then $A_x \in \mathscr{N}$ and $A_y \in \mathscr{M}$ for all $x \in X$ and $y \in Y$.
        \end{theorem}

        \begin{theorem}
            If real function $f$ is $\mathscr{M} \otimes \mathscr{N}$-measurable, then $f_x$ is $\mathscr{N}$-measurable and $f_y$ is $\mathscr{M}$-measurable.
        \end{theorem}

        \begin{theorem}
            If $\mu$ and $\nu$ are complete, then $\mu \times \nu$ are almost never complete.
        \end{theorem}

        \begin{theorem}
            $\mathscr{B}_{\R^n} = \mathscr{B}_{\R} \otimes \mathscr{B}_{\R} \otimes \cdots \otimes \mathscr{B}_{\R}$, that is, the product of Borel measures on $\R$ is the Borel measure on $\R^n$. In general, $\mathscr{B}_{\R^{m \times n}} = \mathscr{B}_{\R^m} \otimes \mathscr{B}_{\R^n}$.
        \end{theorem}

        \begin{theorem}[The Completion]
            $\mathscr{L}_{\R^{m \times n}} = \overline{\mathscr{L}_{\R^m} \otimes \mathscr{L}_{\R^n}}$.
        \end{theorem}

        \begin{theorem}
            Let $X = [0, 1], Y = [0, 1]$, $\mu$ is the Lebesgue measure, and $\nu$ is the counting measure (which is not $\sigma$-finite). There are more than one product measure.
        \end{theorem}

        \subsection{Fubini-Tonelli Theorem}
        \begin{definition}[Monotone Class]
            A collection of subsets $\mathscr{C} \subseteq \mathscr{P}(X)$ is called a monotone class if it is closed under countable increasing unions and countable decreasing intersections, that is,
            \begin{align*}
                E_i \in \mathscr{C}, E_1 \subseteq E_2 \subseteq \cdots \implies \bigcup_i E_i \in \mathscr{C}, \\
                E_i \in \mathscr{C}, E_1 \supseteq E_2 \subseteq \cdots \implies \bigcap_i E_i \in \mathscr{C}.
            \end{align*}
        \end{definition}

        \begin{theorem}
            Let $\mathscr{A}$ be an algebra. Then the monotone class $\mathscr{C}$ generated by $A$ is the $\sigma$-algebra $\mathscr{M}$ generated by $\mathscr{A}$.
        \end{theorem}

        \begin{theorem}[Fubini's Theorem for Simple Functions]
            Let $(X, \mathscr{M}, \mu)$ and $(Y, \mathscr{N}, \nu)$ be $\sigma$-finite measure spaces and $A \in \mathscr{M} \otimes \mathscr{N}$. Then, $\nu(A_x)$ and $\mu(A_y)$ are measurable on $X$ and $Y$, and
            \begin{equation*}
                \int_{X \times Y} \mathbf{1}_A d(\mu \times \nu) = (\mu \times \nu)(A) = \int_X \nu(A_x) d\mu(x) = \int_Y \mu(A_y) d\nu(x).
            \end{equation*}
        \end{theorem}

        \begin{theorem}[Tonelli's Theorem]
            Let $(X, \mathscr{M}, \mu)$ and $(Y, \mathscr{N}, \nu)$ be $\sigma$-finite measure spaces. If $f$ is nonnegative and measurable on $X \times Y$, then
            \begin{equation*}
                g(x) = \int_Y f_x(y) d\nu(y) \;\text{and}\; h(y) = \int_X f_y(x) d\mu(x)
            \end{equation*}
            are measurable on $X$ and $Y$, respectively, and
            \begin{equation*}
                \int_{X \times Y} f d(\mu \times \nu) = \int_X \left(\int_Y f(x, y) d\nu(y)\right) d\mu(x) = \int_Y \left(\int_X f(x, y) d\mu(x)\right) d\nu(y).
            \end{equation*}
        \end{theorem}

        \begin{theorem}[Fubini's Theorem]
            Let $(X, \mathscr{M}, \mu)$ and $(Y, \mathscr{N}, \nu)$ be $\sigma$-finite measure spaces. If $f(x, y)$ is $\mu \times \nu$ integrable, then $f_x(y)$ is $\nu$-integrable for almost every $x$, and $f_y(x)$ is $\mu$-integrable for almost every $y,$ the almost everywhere defined functions
            \begin{equation*}
                g(x) = \int_Y f_x(y) d\nu(y) \;\text{and}\; h(y) = \int_X f_y(x) d\mu(x)
            \end{equation*}
            are integrable with respect to $\mu$ and $\nu$, respectively, and
            \begin{equation*}
                \int_{X \times Y} f d(\mu \times \nu) = \int_X \left(\int_Y f(x, y) d\nu(y)\right) d\mu(x) = \int_Y \left(\int_X f(x, y) d\mu(x)\right) d\nu(y).
            \end{equation*}
        \end{theorem}

        \begin{theorem}[Fubini's Theorem]
            Let $(X, \mathscr{M}, \mu)$ and $(Y, \mathscr{N}, \nu)$ be $\sigma$-finite measure spaces. A measurable $f(x, y)$ is integrable if and only if either one of the following is finite
            \begin{equation*}
                \int_X \left(\int_Y |f_x(y)| d\nu(y)\right) d\mu(x), \quad \int_Y \left(\int_X |f_y(x)| d\mu(x)\right) d\nu(y).
            \end{equation*}
            In this case,
            \begin{equation*}
                \int_{X \times Y} f d(\mu \times \nu) = \int_X \left(\int_Y f(x, y) d\nu(y)\right) d\mu(x) = \int_Y \left(\int_X f(x, y) d\mu(x)\right) d\nu(y).
            \end{equation*}
        \end{theorem}
        
        \subsection{Presentation Question}
        \begin{question}
            Let $\N$ be the set of positive integers, and $\mu$ be the counting measure over $\mathscr{P}(\N)$. Show that a function $f : \N \rightarrow \R$ is measurable and $f$ is integrable over $\N$ if and only if $\sum^{\infty}_{n = 1} f(n)$ is absolutely convergent in which case
            \begin{equation*}
                \int_{\N} f d\mu = \sum^{\infty}_{n = 1} f(n).
            \end{equation*}
        \end{question}
        \begin{proof}
            Let $f : \N \rightarrow \R$. Notice that for any $B \subseteq \R$, $f^{-1}(B) \subseteq N$, which is countable. So $f$ is always measurable.
            
            \noindent \textit{Proof of Sufficiency.} Suppose $f$ is integrable over $\N$. Since each point $n \in \N$ has $\mu(n) = 1$, it follows that
            \begin{equation*}
                \int_{\N} f d\mu = \sum^{\infty}_{n = 1} f(n) \mu(n) = \sum^{\infty}_{n = 1} f(n) \cdot 1 = \sum^{\infty}_{n = 1} f(n).
            \end{equation*}
            Consider the decomposition $f = f^+ - f^-$, where both $f^+$ and $f^-$ are nonnegative. Since $f$ is integrable, by definition,
            \begin{equation*}
                \int_{\N} f^+ d\mu < +\infty \;\text{and}\; \int_{\N} f^- d\mu < +\infty.
            \end{equation*}
            It follows that
            \begin{equation*}
                \sum^{\infty}_{n = 1} |f(n)| = \int_{\N} |f| d\mu = \int_{\N} f^+ + f^- d\mu = \int_{\N} f^+ d\mu + \int_{\N} f^- d\mu < +\infty.
            \end{equation*}
            So $\sum^{\infty}_{n = 1} f(n)$ is absolutely convergent.

            \noindent \textit{Proof of Necessity.} Suppose $\sum^{\infty}_{n = 1} f(n)$ is absolutely convergent in which case
            \begin{equation*}
                \int_{\N} f d\mu = \sum^{\infty}_{n = 1} f(n).
            \end{equation*}
            This implies
            \begin{equation*}
                \sum^{\infty}_{n = 1} |f(n)| = \int_{\N} |f| d\mu = \int_{\N} f^+ + f^- d\mu = \int_{\N} f^+ d\mu + \int_{\N} f^- d\mu < +\infty.
            \end{equation*}
            Since both $f^+$ and $f^-$ are nonnegative, both
            \begin{equation*}
                \int_{\N} f^+ d\mu < +\infty \;\text{and}\; \int_{\N} f^- d\mu < +\infty.
            \end{equation*}
            So $f$ is integrable by definition.
        \end{proof}

        \begin{question}
             Apply Fubini’s theorem to the counting measure on $\N \times \N$ to show that if
             \begin{equation*}
                 \sum^{\infty}_{m = 1} \left(\sum^{\infty}_{n = 1} |a_{mn}|\right) < \infty, \quad a_{mn} \in \R,
             \end{equation*}
             then
             \begin{equation*}
                 \sum^{\infty}_{m = 1} \left(\sum^{\infty}_{n = 1} a_{mn}\right) = \sum^{\infty}_{n = 1} \left(\sum^{\infty}_{m = 1} a_{mn}\right).
             \end{equation*}
        \end{question}
        \begin{proof}
            Consider $f: \N \times \N \rightarrow \R$ such that
            \begin{equation*}
                f(m, n) := a_{mn}, \quad m, n \in \N.
            \end{equation*}
            There is a product measure space $(\N \times \N, P(\N \times \N), \mu \times \mu)$. It follows that
            \begin{equation*}
                \int_{\N \times \N} f(m, n) d(\mu \times \mu) = \sum^{\infty}_{m = 1} \sum^{\infty}_{n = 1} f(m, n) \cdot (\mu \times \mu)(m, n) = \sum^{\infty}_{m = 1} \sum^{\infty}_{n = 1} f(m, n).
            \end{equation*}
            Note that the measure of each set in each term is $1$ since there is only one element in the set. This implies
            \begin{align*}
                \sum^{\infty}_{m = 1} \sum^{\infty}_{n = 1} a_{mn} &< \infty \\
                \sum^{\infty}_{m = 1} \sum^{\infty}_{n = 1} a_{mn} &= \sum^{\infty}_{m = 1} \sum^{\infty}_{n = 1} |f(m, n)| \\
                &= \int_{\N \times \N} |f(m, n)| d(\mu \times \mu) < \infty \\
                \implies f(m, n) &\in L^{1}(\mu \times \mu) \\
                \int_{\N \times \N} f(m, n) d(\mu \times \mu) &= \int_{\N} \left(\int_{\N} f(m, n) d\mu(n)\right) d\mu(m) \\
                &= \int_{\N} \left(\int_{\N} f(m, n) d\mu(m)\right) d\mu(n)
            \end{align*}
            We also have
            \begin{align*}
                \sum^{\infty}_{m = 1} \sum^{\infty}_{n = 1} f(m, n) d(\mu \times \mu) &= \sum^{\infty}_{m = 1} \left(\sum^{\infty}_{n = 1} f(m, n) \mu(\{n\}) \right) \mu(\{m\}) \\
                &= \sum^{\infty}_{n = 1} \left(\sum^{\infty}_{m = 1} f(m, n) \mu(\{m\}) \right) \mu(\{n\}).
            \end{align*}
            By definition of counting measure, for all $m, n \in \N$,
            \begin{equation*}
                \mu(\{m\}) = \mu(\{n\}) = 1.
            \end{equation*}
            Thus, we have
            \begin{equation*}
                \sum^{\infty}_{m = 1} \left(\sum^{\infty}_{n = 1} f(m, n)\right) = \sum^{\infty}_{n = 1} \left(\sum^{\infty}_{m = 1} f(m, n)\right) \implies \sum^{\infty}_{m = 1} \left(\sum^{\infty}_{n = 1} a_{mn}\right) = \sum^{\infty}_{n = 1} \left(\sum^{\infty}_{m = 1} a_{mn}\right).
            \end{equation*}
        \end{proof}

        \begin{question}
            Let
            \begin{equation*}
                f(m, n) = \begin{cases}
                    1 & m = n \\
                    -1 & m = n + 1 \\
                    0 & \text{otherwise}
                \end{cases}
            \end{equation*}
            and $\mu$ is the counting measure over $\N$. Show that $f$ is not integrable over $\N \times \N$ with respect to $\mu \times \mu$.
        \end{question}
        \begin{proof}
            By definition of counting measure, for measurable $E \subseteq \N$,
            \begin{equation*}
                \mu(E) = \begin{cases}
                    |E| & \text{if}\;E\;\text{is finite}, \\
                    \infty & \text{otherwise}
                \end{cases}.
            \end{equation*}
            However, observe that
            \begin{align*}
                \int_{\N \times \N} |f| d\mu d\mu &= \int_{\N} \int_{\N} |f(m, n)| d\mu d\mu \\
                &= \sum^{\infty}_{n = 1} \sum^{\infty}_{m = 1} |f(m, n)| \\
                &= \sum^{\infty}_{n = 1} \sum^{n + 1}_{m = n} |f(m, n)| \cdot 1 \\
                &= \sum^{\infty}_{n = 1} 2 \\
                &= \infty.
            \end{align*}
        \end{proof}


    \section{Lebesgue-Radon-Nikodym Theorem}
        \subsection{Signed Measures}
        \begin{definition}[Signed Measure]
            Let $(X, \mathscr{M})$ be a measurable space. A set function $\nu : \mathscr{M} \rightarrow (-\infty, +\infty]$ or $[-\infty, +\infty)$ is a signed measure if
            \begin{enumerate}
                \item $\nu(\emptyset) = 0$;
                \item $\nu\left(\bigcup^{\infty}_{i = 1} E_i\right) = \sum^{\infty}_{i = 1} \nu(E_i)$ for disjoint $E_i \in \mathscr{M}$.
            \end{enumerate}
        \end{definition}
        
        \begin{definition}[Signed Sets]
            Let $\mu$ be a signed measure. $A \in \mathscr{M}$ is called positive if for any subset $B \subseteq A$, $\mu(B) \geq 0$. It is called negative if $\nu(B) \leq 0$ for any $B \subseteq A$. 
        \end{definition}

        \begin{theorem}[Hahn's Lemma]
            Let $\nu$ be a signed measure and $E$ a measurable set. If $\nu(E) > 0$, then there is a positive set $A \subseteq E$ with $\mu(A) > 0$.
        \end{theorem}
        \begin{proof}
            Let $n_1$ be the smallest integer so tha there is $E_1$ such that $E_1 \subseteq$ and $\mu(E_1) < -\frac{1}{n_1}$. Consider $E_2 \subseteq E \backslash E_1$. If $\mu(E_2) > 0$, done. Otherwise continue the process. In this case, let
            \begin{equation*}
                A := E \backslash \bigcup^{\infty}_{i = 1} E_i.
            \end{equation*}
            Observe that
            \begin{equation*}
                \mu(A) = \mu(E) - \mu\left(\bigcup^{\infty}_{i = 1} E_i\right) > \mu(E) > 0.
            \end{equation*}
            This implies if $B \subseteq A$,
            \begin{equation*}
                B \subseteq E \backslash \bigcup^{\infty}_{i = 1} E_i \subseteq E \backslash \bigcup^{k}_{i = 1} E_i, \quad \forall k.
            \end{equation*}
            Then
            \begin{equation*}
                \mu(B) \geq -\frac{1}{n_k - 1}, n_k \rightarrow 0 \implies \mu(B) \geq 0.
            \end{equation*}
        \end{proof}

        \begin{theorem}[Hahn's Decomposition Theorem]
            Let $\nu$ be a signed measure in $(X, \mathscr{M})$. There exists a positive set $A$ so that $A^C$ is negative.
        \end{theorem}

        \begin{definition}[Concentration of Measure]
            Measure $\mu$ is said to be concentrated on a set $A \in \mathscr{M}$ if for all $E \in \mathscr{M}$,
            \begin{equation*}
                \mu(E) = \mu(E \cap A).
            \end{equation*}
        \end{definition}

        \begin{definition}[Mutually Singular Measures]
            Two measures $\mu$ and $\nu$ in $(X, \mathscr{M})$ are said mutually singular if there are disjoint sets $A, B \in \mathscr{M}$ so that $\mu$ is concentrated on $A$ and $\nu$ is concentrated on $B$. Write $\mu \perp \nu$. It means that $\mu$ and $\nu$ concentrate on disjoint sets.
        \end{definition}

        \begin{definition}[Jordan's Decomposition]
            There exists unique measures $\mu^+$ and $\mu^-$ so that
            \begin{equation*}
                \mu = \mu^+ - \mu^-,
            \end{equation*}
            where there are mutually singular and one of them is finite.
        \end{definition}

        \subsection{The Lebesgue-Radon-Nikodym Decomposition Theorem}
        \begin{theorem}[Absolutely Continuous]
            $\nu$ is absolutely continuous with respect to $\mu$ if $\mu(E) = 0$ implies $\nu(E) = 0, E \in \mathscr{M}$.
        \end{theorem}

        \begin{theorem}
            Let $f(x) \geq 0$ be measurable. Define
            \begin{equation*}
                \nu(E) = \int_E f(x) d\mu(x).
            \end{equation*}
            Then $\nu$ is a measure and $\nu \ll \mu$.
        \end{theorem}

        \begin{theorem}[Radon-Nikodym Theorem]
            Let $(X, \mathscr{M}, \mu)$ be a $\sigma$-finite and $\nu$ be $\sigma$-finite on $\mathscr{M}$ that is absolutely continuous with respect to $\mu$. Then,
            \begin{equation*}
                \nu(E) = \int_E f d\mu, \quad E \in \mathscr{M},
            \end{equation*}
            where $f$ is nonnegative and measurable that is unique in the sense of $\mu$ almost everywhere.
        \end{theorem}
        \begin{remark}
            It is also written
            \begin{equation*}
                d\nu = f d\mu, \quad \text{and}\; \frac{d\nu}{d\mu} = f.
            \end{equation*}
            is also called called Radon-Nikodym derivative.
        \end{remark}

        \begin{theorem}[Lebesgue-Radon-Nikodym Decomposition]
            Let $(X, \mathscr{M}, \mu)$ be $\sigma$-finite. For any measure $\nu$ on $\mathscr{M}$, there exist unique measures $\nu_c$ and $\nu_s$ such that
            \begin{equation*}
                \nu = \nu_c + \nu_s = \int f d\mu + \nu_s, \quad \nu_c \ll \mu, \quad \nu_s \perp \mu.
            \end{equation*}
        \end{theorem}
        \begin{remark}
            May not be true if not $\sigma$-finite.
        \end{remark}
        
        \subsection{Presentation Question}
        \begin{question}
            Let $\mu_1, \dots, \mu_n, \nu$ be measures in measurable spaces $(X, \mathscr{M})$ and
            \begin{equation*}
                \mu = \sum^{n}_{j = 1} c_j \mu_j, \quad c_j > 0.
            \end{equation*}
            Show that $\mu_j \ll \mu$, and if $\mu_j \perp \nu$, show that $\mu \perp \nu$.
        \end{question}
        \begin{proof}
             % \ll absolutely continuous
             Consider the case when $\mu(E) = 0$.
             \begin{equation*}
                 0 = \mu(E) = \sum^{n}_{j = 1} c_j \mu_j(E).
             \end{equation*}
             Since $c_j > 0$ and $\mu_j(E) \geq 0$, it follows that for all $1, 2, \dots, n$,
             \begin{equation*}
                 c_j \mu_j(E) \geq 0, \implies c_j \mu_j(E) = 0
             \end{equation*}
            Thus, for any $i = 1, 2, \dots, n$, $\mu_j(E) = 0$ and $\mu_j \ll \mu$.

            \noindent Suppose $\mu_j \perp \nu$ for all $j \in \{1, 2, \dots, n\}$. Suppose $\nu$ is concentrated on set $E$, and $\mu_j$ is concentrated on $F_j$. By definition of mutually singular sets, for any $j \in \{1, 2, \dots, n\}$, $\mu_{j}(A) = 0$ for any $A \subseteq E$. This implies
            \begin{equation*}
                \mu(A) = \sum^{n}_{j = 1} c_j \mu_j(E) = \sum^{n}_{j = 1} 0 = 0 = \mu\left(A \cap \bigcup^{n}_{j = 1} F_j\right).
            \end{equation*}
            Similarly, for any $B \subseteq \bigcup^{n}_{j = 1} F_j$, if $B \cap E \neq \emptyset$, there exists some $j \in \{1, 2, \dots, n\}$ such that $B_j \subseteq F_j$. It follows that for this $j$,
            \begin{equation*}
                F_j \cap E \neq \emptyset \implies \mu_j \not\perp \nu,
            \end{equation*}
            which is a contradiction. This implies
            \begin{equation*}
                E \cap \bigcup^{n}_{j = 1} F_j = \emptyset \implies \mu \perp \nu.
            \end{equation*}
        \end{proof}

        \begin{question}
            Suppose $F(x)$ is differentiable in $\R$ and $F'(x)$ is nonnegative and Riemann integrable in every interval $[a, b]$, $a < b$, $a, b \in \R$. Show that $dm_F = F' dm$.
        \end{question}
        \begin{proof}
            Since $F'(x)$ is Riemann integrable, by fundamental theorem of calculus,
            \begin{equation*}
                F(b) - F(a) = \int^b_a F'(x) dm(x).
            \end{equation*}
            $F'(x)$ is non-negative implies $F(x)$ is increasing and continuous. So there is a unique Borel measure $m_F$ such that
            \begin{equation*}
                m_F((a, b]) = F(b) - F(a) = \int^b_a F'(x) dm(x).
            \end{equation*}
            So $m_F$ is also absolutely continuous with respect to $m$. Thus, by definition of the Radon-Nikodym Derivative, this implies $dm_F = F' dm$.
        \end{proof}


    \section{$L^p$ Spaces}
        \subsection{Normed Vector Spaces}
        \begin{definition}[Norm]
            A norm of a vector space is a function $\|\cdot\| \rightarrow \R$ satisfying:
            \begin{enumerate}
                \item $\|x\| \geq 0$, and $\|x\| = 0$ if and only if $x = 0$ (positivity).
                \item $\|a \cdot x\| = |a| \|x\|, a \in \R, x \in X$ (homogeneity).
                \item $\|x + y\| \leq \|x\| + \|y\|$ (triangular inequality).
            \end{enumerate}
        \end{definition}

        \begin{definition}[Normed Vector Space]
            $(X, \|\cdot\|)$ is a vector space with a norm.
        \end{definition}
        
        \subsection{Banach Spaces}
        \begin{definition}[Convergence]
            A sequence $\{x_n\}$ in a normed vector space $(X, \|\cdot\|)$ is said $z \in X$ if $\|x_n - z\| \rightarrow 0$ as $n \rightarrow \infty$.
        \end{definition}
        \begin{remark}
            In $\left(C([a, b]), \|\cdot\|_{\max}\right)$ and $\left(B(E), \|\cdot\|_{\max}\right)$, the convergence is uniform convergence. \\
            In $\left(L^1(E), \|\cdot\|_{1}\right)$, the convergence $\|f_n - f\|_1 = \int_E |f_n - f| d\mu \rightarrow 0$ is $L^1$ convergence.
        \end{remark}

        \begin{definition}
            A sequence $\{x_n\}$ in $X$ is called a Cauchy sequence if for any $\varepsilon > 0$ there is $N > 0$ so that whenever $n, m \geq N$,
            \begin{equation*}
                \|x_n - x_m\| < \varepsilon.
            \end{equation*}
        \end{definition}
        \begin{remark}
            Any convergent sequence is cauchy, but the reverse may not be true.
        \end{remark}

        \begin{definition}[Completeness]
            A normed vector space $X$ is called complete if every Cauchy sequence is convergent.
        \end{definition}

        \begin{definition}[Banach Space]
            A Banach space is a complete normed vector space.
        \end{definition}

        \subsection{$L^p$ Spaces}
        \begin{definition}[$L^p$ Norm]
            Let $(X, \mathscr{M}, \mu)$ be a measure space and $E \in \mathscr{M}$. For a measurable function, $f : E \rightarrow \R$, define the $L^p$-norm by
            \begin{equation*}
                \|f\|_p = \left(\int_E |f|^p d\mu\right)^{\frac{1}{p}}, \quad p \geq 1.
            \end{equation*}
        \end{definition}

        \begin{definition}[$L^p$ Space]
            Define the $L^p$ space
            \begin{equation*}
                L^p(E, \mu) = \{f : E \rightarrow \R : \|f\|_p < \infty\}.
            \end{equation*}
        \end{definition}

        \begin{theorem}
            $L^p(E, \mu)$ is a normed vector space when $p \geq 1$.
        \end{theorem}

        \begin{theorem}[H\"{o}lder's Inequality]
            Let $1 < p < \infty$ and $\frac{1}{p} + \frac{1}{q} = 1$. If $f, g \in L^p(E, \mu)$ then
            \begin{equation*}
                \|fg\|_1 \leq \|f\|_p \|g\|_q,
            \end{equation*}
            with equality if and only of $a |f|^p = b |g|^q$ almost everywhere for some non-zero constants $a, b$.
        \end{theorem}

        \begin{theorem}[Minkowski Inequality]
            Let $1 < p < \infty$. If $f, g \in L^p(E, \mu)$ then
            \begin{equation*}
                \|f + g\|_p \leq \|f\|_p + \|g\|_p,
            \end{equation*}
            with equality for $p > 1$ if and only if $f(x) = bg(x)$ for some $b > 0$, $f(x) = 0$, or $g(x) = 0$ almost everywhere.
        \end{theorem}

        \begin{theorem}
            The normed vector space $L^p(E, \mu)$ is a Banach space.
        \end{theorem}

        
\end{document}