\documentclass[class=book, crop=false]{standalone}
\usepackage{xcolor}
\usepackage{utils}
\usepackage{import}

\begin{document}
    \section{General Measures, Outer Measures, Carath\'{e}odory Theorem}
        \subsection{General Measure}
        \begin{definition}[Measure]
            Let $(X, \mathscr{M})$ be a measurable space ($\mathscr{M}$ is a $\sigma$-algebra of $X$). A set function $\mu : \mathscr{M} \rightarrow [0, +\infty]$ if it satsifies:
            \begin{enumerate}
                \item $\mu(\emptyset) = 0$.
                \item ($\sigma$-additivity, countable additivity) for any sequence of mutually disjoint sequence $E_i \in \mathscr{M}$, there is
                \begin{equation*}
                    \mu\left(\bigcup^{\infty}_{i = 1} E_i\right) = \sum^{\infty}_{i = 1} \mu(E_i).
                \end{equation*}
            \end{enumerate}
        \end{definition}
        \begin{remark}
            $\mu$ is called finite measure if $\mu(X) < \infty$, probability measure if $\mu(X) = 1$.
        \end{remark}

        \begin{definition}[Measure Space]
            If $\mu$ is a measure in measurable space $(X, \mathscr{M})$, then $(X, \mathscr{M}, \mu)$ is called a measure space.
        \end{definition}
        \begin{remark}
            Probability space if $\mu(X) = 1$.
        \end{remark}

        \noexpand Let $(X, \mathscr{M}, \mu)$ be a measure space.
        \begin{theorem}[Finite additivity]
            If $E_1, E_2, \dots, E_n \in \mathscr{M}$ were disjoint, then $\mu\left(\bigcup^{\infty}_{i = 1} E_i\right) = \sum^{\infty}_{i = 1} \mu(E_i)$.
        \end{theorem}
        
        \begin{theorem}[Monotonicity]
            If $E_1 \subseteq E_2$, then $\mu(E_1) \leq \mu(E_2)$.
        \end{theorem}
        
        \begin{theorem}[Sub-additivity]
            For $E_1, E_2, \dots, E_n \in \mathscr{M}$, $\mu\left(\bigcup^{\infty}_{i = 1} E_i\right) \leq \sum^{\infty}_{i = 1} \mu(E_i)$.
        \end{theorem}

        \begin{theorem}[Continuity from Below]
            If $E_1 \subseteq E_2 \subseteq \cdots$, then $\mu\left(\bigcup^{\infty}_{i = 1} E_i\right) = \lim_{i \rightarrow \infty} \mu(E_i)$.
        \end{theorem}

        \begin{theorem}[Continuity from Above]
            If $E_1 \subseteq E_2 \subseteq \cdots$ and $\mu(E_1) < \infty$, then $\mu\left(\bigcap^{\infty}_{i = 1} E_i\right) = \lim_{i \rightarrow \infty} \mu(E_i)$.
        \end{theorem}

        \begin{theorem}[Lower Continuity]
            $\mu\left(\varliminf_{i \rightarrow \infty} E_i\right) \leq \varliminf_{i \rightarrow \infty} \mu\left(E_i\right)$
        \end{theorem}

        \begin{theorem}[Upper Continuity]
            If $\mu\left(\bigcup^{\infty}_{i = 1} E_i\right) < \infty$, then $\mu\left(\varlimsup_{i \rightarrow \infty} E_i\right) \geq \varlimsup_{i \rightarrow \infty} \mu\left(E_i\right)$.
        \end{theorem}

        \begin{theorem}
            If $\lim_{i \rightarrow \infty} E_i$ exists and $\mu\left(\bigcup^{\infty}_{i = 1} E_i\right) < \infty$, then $\mu\left(\lim_{i \rightarrow \infty} E_i\right) = \lim_{i \rightarrow \infty} \mu\left(E_i\right)$.
        \end{theorem}

        \begin{theorem}[Subtraction]
            If $E \subseteq F$ and $\mu(E) < \infty$, then $\mu(F \backslash E) = \mu(F) - \mu(E)$.
        \end{theorem}

        \begin{theorem}[The Borel-Cantelli Lemma]
            If $\sum^{\infty}_{i = 1} \mu(E_i) < \infty$, then $\mu\left(\varlimsup_{i \rightarrow \infty} E_i\right) = 0$, that is, the set of elements that belongs to infinitely many $E_i$ has measure zero.
        \end{theorem}

        \begin{definition}[Finite Measure]
            A measure $\mu$ in a measurable space $(X, \mathscr{M})$ is called finite if $\mu(X) < \infty$.
        \end{definition}
        \begin{remark}
            Probability and discrete measures is finite; the Lebesgue measure is not finite
        \end{remark}

        \begin{definition}[$\sigma$-Finite]
            A measure $\mu$ in a measurable space $(X, \mathscr{M})$ is called finite if
            \begin{equation*}
                X = \bigcup^{\infty}_{i = 1} X_i, \mu(X_i) < \infty.
            \end{equation*}
        \end{definition}
        \begin{remark}
            Lebesgue measure is $\sigma$-finite.
        \end{remark}

        \begin{definition}[Restriction]
            Let $(X, \mathscr{M}, \mu)$ be a measure space and $A \in \mathscr{M}$. Define function $\mu_A$ in $(X, \mathscr{M})$ by
            \begin{equation*}
                \mu_A(E) = \mu(E \cap A).
            \end{equation*}
            Then $\mu_A$ is a measure in $(X, \mathscr{M})$, $\mu_A : \mathscr{M} \rightarrow [0, +\infty]$. Note that $\mu_A(E \cap A^C) = 0$. $(A, M|_A, \mu_A)$ forms a measure space, where $M|_A = \{E \cap A : E \in \mathscr{M}\}$.
        \end{definition}

        \begin{definition}[Complete Measure Space]
            A measure space $(X, \mathscr{M}, \mu)$ is complete if any subset of a set of measure $0$ is also measurable.
        \end{definition}
        \begin{remark}
            Lebesgue measure space is complete. Borel measure space is not.
        \end{remark}

        \begin{theorem}
            Let $(X, \mathscr{M}, \mu)$ be a measure space, and
            \begin{equation*}
                \mathscr{N} = \{N \in M : \mu(N) = 0\}, \quad \bar{M} = \{E \cup F : E \in M, F \subseteq N, \exists N \in \mathscr{N}\}.
            \end{equation*}
            Then $\bar{M}$ is a $\sigma$-algebra, there is a unique extension $\bar{\mu}$ of $\mu$ to a complete measrue on $\bar{\mathscr{M}}$. \\
            $\bar{\mu}$ is the completion of $\mu$. It is defined by $\bar{\mu}(E \cup F) = \mu(E)$ for $E \cup F \in \bar{\mathscr{M}}$ and $F \subseteq N \in \mathscr{N}$.
        \end{theorem}

        \subsection{Outer Measure and its Construction}
        \begin{definition}[Outer Measure]
            Let $X \neq \emptyset$. A set function $\mu^* : \mathscr{P}(X) \rightarrow [0, \infty]$ is called an outer measure if
            \begin{itemize}
                \item $\mu^*(\mathscr{P}) = \emptyset$;
                \item $\mu^*(A) \leq \mu^*(B)$ when $A \subseteq B$ (monotonicity);
                \item $\mu^*(\bigcup^{\infty}_{i = 1} A_i) \leq \sum^{\infty}_{i = 1} \mu^*(A_i)$ when $A_i \in \mu^*(\mathscr{P}(X))$ (subadditivity).
            \end{itemize}
        \end{definition}
        \begin{remark}
            Outer measure is defined on the power set.
        \end{remark}

        \begin{theorem}
            Let $\mathscr{E}$ be a family of subsets of $X$ including $\emptyset$ and $X$. Let $\rho : \mathscr{E} \rightarrow [0, \infty]$ be a set function such that $\rho(\emptyset) = 0$. For any $E \in \mathscr{E}$ define the set function $\mu^* : \mathscr{P}(X) : \mathscr{E} \rightarrow [0, \infty]$ by
            \begin{equation*}
                \mu^*(E) = \inf\left\{\sum^{\infty}_{i = 1} \rho(E_i) : E \subseteq \bigcup^{\infty}_{i = 1} E_i, E_i \in \mathscr{E}\right\}.
            \end{equation*}
            Then $\mu^*$ is an outer measure.
        \end{theorem}

        \begin{definition}[Metric Outer Measure]
            Outer measure $\mu^*$ is called an metric outre measure if for any sets $E, F$ with $d(E, F) := \inf\{d(x, y) : x \in E, y \in E\} > 0$ it always have $\mu^*(E \cup F) = \mu^*(E) + \mu^*(F)$.
        \end{definition}

        \subsection{Carath\'eodory's Theorem}
        \begin{theorem}[$\mu^*$-Measurable]
            Let $\mu^*$ be the outer measure in $X$. A subset $E \subseteq X$ is called $\mu^*$-measurable if
            \begin{equation*}
                \mu^*(A) = \mu^*(A \cap E) + \mu^*(A \cap E^C), \quad\text{for any}\; A \subseteq X.
            \end{equation*}
        \end{theorem}

        \begin{remark}
            By subadditivity, it is always true that $\mu^*(A) \leq \mu^*(A \cap E) + \mu^*(A \cap E^C)$. So $E$ is measurable if and only if $\mu^*(A) \geq \mu^*(A \cap E) + \mu^*(A \cap E^C)$.
        \end{remark}

        \begin{theorem}
            $\mu^*$ is finitely additive on $\mu^*$-measurable sets.
        \end{theorem}

        \begin{theorem}[Carath\'eodory Theorem]
            Let $\mu^*$ be an outer measure in $X$ and let $\mathscr{M}$ be the family of $\mu^*$-measurable sets. Then $\mathscr{M}$ is a $\sigma$-algebra and the restriction of $\mu^*$ to $\mathscr{M}$ is a complete measure.
        \end{theorem}
        
        \subsection{Presentation Question}
        \begin{question}
            Show that $p$-dim Hausdorff outer measure then $H_p$ is a metric outer measure.
        \end{question}
        
        \begin{definition}[Diameter]
            Diameter of region $E$ is $d(E) = \sup\{d(x, y) : x, y \in E\}$.
        \end{definition}

        % \leq follows from subadditivity
        % follow the same logic
        % \sum{\infty}_{i = 1} D(A_i) \geq H^p_{\delta}(E) + H^p_{\delta}(F) => complete
        \begin{proof}
            Let $S_A := \bigcup^{\infty}_{i = 1} A_i, S_B := \bigcup^{\infty}_{i = 1} B_i$ be the union covers of sets $A, B \subseteq X$, respectively. Suppose $d(A, B) > 0$. Let $S_{A \cup B} := \bigcup^{\infty}_{i = 1} C_i$ be a cover of $A \cup B$. Without loss of generality, assume each $C_i \cap (A \cup B) \neq \emptyset$. Notice that when $\delta < \frac{1}{3} d(A, B)$, for any $i \in \Z^+$, at least one of $C_i \cap A$ and $C_i \cap B$ must be empty. So $\{C_i\}$ can be partitioned into $2$ disjoint covers such that $A \subseteq \bigcup^{\infty}_{j = 1} C_{i_j}$ and $B \subseteq \bigcup^{\infty}_{k = 1} C_{i_k}$, that is,
            \begin{align*}
                \mu^*(A \cup B) &= \mu^*(A) + \mu^*(B) \\
                H^p_{\delta}(A \cup B) &= H^p_{\delta}(A) + H^p_{\delta}(B) \\
                \lim_{\delta \rightarrow 0} H^p_{\delta}(A \cup B) &= \lim_{\delta \rightarrow 0} H^p_{\delta}(A) + \lim_{\delta \rightarrow 0} H^p_{\delta}(B) \\
                H^p(A \cup B) &= H^p(A) + H^p(B).
            \end{align*}
            So $H^p$ is a metric outer measure.
        \end{proof}


    \section{Premeasure, Lebesgue-Stieltjes and Push-Forward Measures}
        \subsection{Premeasure}
        \begin{definition}[Premeasure]
            Let $\mathscr{A}$ be an algebra of $X$. A set function $\mu_0 : \mathscr{A} \rightarrow [0, \infty]$ is a premeasure if it satisfies:
            \begin{enumerate}
                \item $\mu_0(\emptyset) = 0$;
                \item $\mu_0\left(\bigcup^{\infty}_{i = 1} A_i\right) = \sum^{\infty}_{i = 1} \mu_0(A_i)$ for disjoint $A_i \in \mathscr{A}$ so that $\bigcup^{\infty}_{i = 1} A_i \in \mathscr{A}$.
            \end{enumerate}
        \end{definition}

        \begin{theorem}
            A premeasure is finitely additive and monotone.
        \end{theorem}

        \begin{theorem}
            Let $\mu_0$ be a measrue on an algebra $\mathscr{A}$. Then for any $A \in \mathscr{A}$, $A$ is $\mu_0$ measurable, i.e., $A \in \mathscr{M}$, and $\mu^*_0(A) = \mu_0(A)$.
        \end{theorem}

        \begin{theorem}[Carath\'eodory Extension Theorem]
            Let $\mathscr{A}$ be an algebra of $X$, $\sigma(\mathscr{A})$ be a $\sigma$-algebra generated by $\mathscr{A}$, and $\mu_0$ be a premeasure on $\mathscr{A}$. Then there exists a measure $\mu$ on $\sigma(\mathscr{A})$ such that
            \begin{equation*}
                \mu = \mu^*_0 |_{\sigma(\mathscr{A}}, \quad \mu(A) = \mu_0(A), \quad A \in \mathscr{A}.
            \end{equation*}
            If $\mu^*_0$ is $\sigma$-finite, then $\mu(\mathscr{A}) \rightarrow [0, \infty]$ is the unique extension of $\mu_0 : \mathscr{A} \rightarrow [0, \infty]$ to $\sigma(\mathscr{A})$.
        \end{theorem}
        
        \subsection{Lebesgue-Stieltjes and Push-Forward Measures}
        \begin{theorem}
            Let $F : \R \rightarrow \R$ be an increasing and right continuous function. For a disjoint union $\bigcup^{\infty}_{i = 1} (a_i, b_i]$, define set function $\mu_0 : \mathscr{A} \rightarrow [0, \infty]$ by
            \begin{equation*}
                \mu_0\left(\bigcup^{\infty}_{i = 1} (a_i, b_i]\right) = \sum^{\infty}_{i = 1} (F(b_i) - F(a_i)).
            \end{equation*}
            Then $\mu^*_0$ is a premeasure on $\mathscr{A}$.
        \end{theorem}

        \begin{definition}[Lebesgue-Stieltjes Measure]
            The premeasure defined above has an extension $\mu_F : \mathscr{M}_F \rightarrow [0, \infty]$, where $\mathscr{M}_F$ is the family of $\mu_F$-measurable sets. $\mu_F$ is called Lebesgue-Stieltjes measure.
        \end{definition}
        
        \begin{remark}
            There is a similar construction for intervals of the form $[a, b)$ and a left continuous function $F$.
        \end{remark}

        \begin{theorem}
            If $\mu$ is a finite Borel measure in $\R$, let $F(x) = \mu((-\infty, x])$, then $\mu = m_F$. $F$ is called cumulative distribution function.
        \end{theorem}

        \begin{theorem}
            The completion of the Borel measure $m_F$ is the restriction of the measure $\mu^*_0$ to the $\mu^*_0$ measurable sets. It is called Lebesgue-Stieltjes measure, denoted by $m_F$.
        \end{theorem}

        \begin{definition}[Push-forward Measure]
            Let $(X, \mathscr{M})$ and $(Y, \mathscr{N})$ be measurable spaces and $f : X \rightarrow Y$ be a measurable function. If $\mu$ is a measure in $(X, \mathscr{M})$, one can define a measure $\nu$ in $(Y, \mathscr{N})$ by
            \begin{equation*}
                \nu(B) = \mu(f^{-1}(B)), \quad B \in \mathscr{N}.
            \end{equation*}
            It is easily verified that $\nu$ is a measure in $(Y, \mathscr{N})$. $\nu$ is the push-forward measure of $\mu$, denoted $f_*\mu$.
        \end{definition}
        
        \subsection{Presentation Question}
        \begin{question}
            Let
            \begin{equation*}
                F(x) = \begin{cases}
                    x + 2 & x \geq 1 \\
                    x^3 & -1 \leq x < 1 \\
                    -2 & x < -1
                \end{cases}
            \end{equation*}
            Find the Lebesgue-Stieltjes measure $m_F$, and compute $m_F([-1, 0])$ and $m_F([-1, 1])$.
        \end{question}
        % Consider b = a + m
        \begin{proof}
            Notice that the function has a jump at $x = -1$ and $x = 1$. Since $m_F = F(b) - F(a)$, the Lebesgue-Stieltjes measure is
            \begin{align*}
                m_F &= m_{-2 | (-\infty, -1)} + \delta_{-1} + m_{x^3 | [-1, 1)} + 2\delta_1 + m_{x + 2 | [1, +\infty)} \\
                &= (-2 - (-2)) + \delta_{-1} + m_{x^3 | [-1, 1)} + 2\delta_1 + m \\
                &= \delta_{-1} + m_{x^3 | [-1, 1)} + 2\delta_1 + m^+.
            \end{align*}
            where $m^+$ is the restriction of Lebesgue measure $m$ to $[1, +\infty)$. \\
            This implies
            \begin{align*}
                m_F([-1, 0]) &= m_F([-1, -1]) + m_F((-1, 0]) \\
                &= F(-1) - F(-1^-) + F(0) - F(-1) \\
                &= F(0) - F(-1^{-}) \\
                &= 0 - (-2) \\
                &= 2; \\
                m_F([-1, 1]) &= m_F([-1, -1]) + m_F((-1, 1]) \\
                &= F(-1) - F(-1^-) + F(1) - F(-1) \\
                &= F(1) - F(-1^{-}) \\
                &= 1 - (-2) \\
                &= 3.
            \end{align*}
        \end{proof}
\end{document}